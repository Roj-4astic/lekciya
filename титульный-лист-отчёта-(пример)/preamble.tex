\documentclass[psamsfonts, intlimits, sumlimits, namelimits, 14pt, a4paper]{extarticle}
\usepackage[left = 3.0cm, right = 1.5cm, top = 2.0cm, bottom = 2.0cm]{geometry} % геометрия страницы, в т.ч. поля

\usepackage{caption} % поддержка подписей рисунков и т.д.

\usepackage[T2A]{fontenc} % распознование кодировки текста в tex-файле
\usepackage[utf8]{inputenc}% распознование шрифтов
\usepackage[english, main=russian]{babel}% пакет поддержки орфографий
\usepackage{tempora} %пакет шрифта Tempora

\usepackage{etex}

% расширенное управление переносом слов, разделенных пунктиром
\usepackage[shortcuts]{extdash}
% поддержка настроек междустрочного интервала
\usepackage{setspace}
% Настройка сносок
\usepackage[perpage,bottom,multiple,stable]{footmisc}

%% Пакеты для гипертекста
\RequirePackage{color}
\usepackage{hyperref} % обширная поддержка гипертекста
\hypersetup{backref,
% colorlinks=false,
linktoc=all
}
\hypersetup{pdfborder=0 0 0}
\hypersetup{pdfencoding=auto}

%% Расширенная математика
\usepackage{amsmath}
\usepackage{amssymb}
\usepackage{amsthm}
\usepackage{amscd}

\usepackage{accents} % пользовательские акценты
\usepackage{cmap} % предоставляет таблицы сопоставления символов для PDF
\usepackage{textcomp} % поддержка шрифтов Text Companion
\usepackage{mathtext} % для "прозрачного" использования кириллических букв в формулах

\usepackage{mathtools}
\mathtoolsset{
showonlyrefs=true, % нумеровать только формулы, на которых есть ссылки
mathic=true,
}

\allowdisplaybreaks

% Поддержка перечислений и списков
\usepackage{paralist}
% \usepackage{enumitem} % не совместим с пакетом paralist
% см. о разнице кратко https://tex.stackexchange.com/questions/18411/what-are-the-differences-between-using-paralist-vs-enumitem

%% Многостраничные таблицы
% \usepackage{longtable}
% \usepackage{lscape}
% \usepackage{makecell}
% \usepackage{multirow}
% \usepackage{tabularx}
% или
\usepackage{tabularray}
\UseTblrLibrary{functional}
\UseTblrLibrary{diagbox}

\DefTblrTemplate{contfoot-text}{russian}{Продолжение на следующей странице}
\SetTblrTemplate{contfoot-text}{russian}
\DefTblrTemplate{conthead-text}{russian}{(Продолжение)}
\SetTblrTemplate{conthead-text}{russian}


%% Работа с графикой
\usepackage{graphicx}

\usepackage{float}
\usepackage{epic}
\usepackage{rotating}

\usepackage[shell, subfolder, cleanup]{gnuplottex}
\def\gnuplotexe{/usr/bin/gnuplot}
% \def\gnuplotexe{C:/Program\ Files/gnuplot/bin/gnuplot.exe}

%% Путь к рисункам
\graphicspath{{images}{gnuplottex}}

%% Обработка текста в eps-файлах
\usepackage[scanall]{psfrag}

%% обработка блок-схем
\usepackage{tikz}
\usetikzlibrary{arrows, shapes, graphs, automata, positioning}

%% Пакеты для библиографии
\usepackage{cite}%
% \usepackage{gost7184}
% \bibliographystyle{gost7184}
% \bibliographystyle{ugost2003s}
\bibliographystyle{gost2008ls}
% \bibliographystyle{utf8gost780u}
%\renewcommand{\bibname}{Литература} % если класс book или report
\renewcommand{\refname}{Литература} % если класс article

%% Пакеты для листингов
%% Расширеное окружение verbatim
\usepackage{fancyvrb}
\usepackage{url}

% Наряду, или вместо verbatim можно использовать более специализированный.
% Пакет listings
\usepackage{listings}
\usepackage{listingsutf8}
\lstset{%
    showstringspaces=false,
    numbers=left,
%    numberstyle=\tiny,
%    upquote=false,
    keepspaces=true,
    columns=flexible,
    basicstyle=\footnotesize\ttfamily,%
    breaklines=true,%
    breakatwhitespace=true,%
    postbreak=\space,%
    prebreak={\mbox{\quad$\hookleftarrow$}},%
}

%\ifthenelse{\boolean{luatex}\OR\boolean{xetex}}{}{%
%    \lstset{inputencoding=utf8/koi8-r}}

\lstloadlanguages{C,make,bash,[x86masm]Assembler,[LaTeX]TeX, [08]Fortran, [95]Fortran, Gnuplot}

\usepackage{indentfirst} % отступ первой строки каждого раздела по умолчанию
\renewcommand\baselinestretch{1.5} % междустроный интервал всего документа



%%% Local Variables:
%%% mode: latex
%%% coding: utf-8-unix
%%% TeX-master: "./default"
%%% End: