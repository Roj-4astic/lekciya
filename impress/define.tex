\ifx\pstmode\undefined
\else
    \setbeamertemplate{navigation symbols}{}
\fi

\ifdefined\withnotes
	\usepackage{pgfpages}

	\ifdefined\onlynotes
	\setbeameroption{show only notes}
	\setbeamertemplate{note page}[compress]
	\pgfpagesuselayout{resize to}[a4paper,border shrink=5mm,landscape]
	\else
	\setbeameroption{show notes}
	\setbeamertemplate{note page}[plain]
	\pgfpagesuselayout{8 on 1}[a4paper,border shrink=5mm]
	\fi

	% Версия для печати
	\ifdefined\printable
	  \usecolortheme{dove}
	  \setbeamertemplate{navigation symbols}{}
	  \setbeamercolor{note page}{bg=white}
	  \setbeamercolor{note title}{bg=white}
	  \setbeamercolor{note date}{bg=white}
	\fi

	% http://tex.stackexchange.com/questions/288408/beamer-show-notes-on-second-screen-with-xelatex-and-atbeginsection
	\ifthenelse{\boolean{xetex}}{%
	  \makeatletter
	  \def\beamer@framenotesbegin{% at beginning of slide
		 \usebeamercolor[fg]{normal text}
		 \gdef\beamer@noteitems{}%
		 \gdef\beamer@notes{}%
	  }
	  \makeatother
	}{}
\else\fi

\ifdefined\handoutwithnotes
	\usepackage{handoutWithNotes}
	\pgfpagesuselayout{4 on 1 with notes}[a4paper,border shrink=5mm]
\else\fi

\ifdefined\poster
	\mode<presentation> {
		 \usetheme{PFU}				% Название темы оформления для постера
		 \setbeamertemplate{navigation symbols}{}	% Убираем навигационные символы
	}

	\usepackage{ragged2e}
	\boldmath			% Математические формулы будут полужирными, чтоб в глаза било

	%\usepackage[orientation=album,size=a0,scale=1.4,debug]{beamerposter}                       	% e.g. for DIN-A0 poster
	%\usepackage[orientation=portrait,size=a1,scale=1.4,grid,debug]{beamerposter}                  	% e.g. for DIN-A1 poster, with optional grid and debug output
	%\usepackage[size=custom,width=200,height=120,scale=2,debug]{beamerposter}                     	% e.g. for custom size poster
	%\usepackage[orientation=portrait,size=a0,scale=1.0,printer=rwth-glossy-uv.df]{beamerposter}   	% e.g. for DIN-A0 poster with rwth-glossy-uv printer check
	%\usepackage[orientation=album,size=a0,scale=1.4]{beamerposter}
	\usepackage[orientation=portrait,size=a1,scale=1.0]{beamerposter}
\else\fi

\usepackage{pifont}

%% настройки геометрии
%\RequirePackage{xkeyval}

\usepackage{bm} % определяет команду \bm{<текст>} дающее полужирное начертание тексту <текст> в аргументе

\usepackage{texnames}

\usepackage{wrapfig} % позволяет обернуть текст вокруг рисунков или таблиц

\def\bibnamefont{\textit} % задаёт курсив Ф.И.О. авторов в списке литературы

\usepackage{calc}
\usepackage{soulutf8} % пакет soul с поддержкой utf8: позволяет быстро форматировать текст, например выделить командой \hl


% \usepackage[nodayofweek]{datetime}
\usepackage[24hr]{datetime}

\nocite{*} %  включает в спиcок литературы все записи что без ссылок в тектсе через \cite

\usepackage{afterpage}
\usepackage{subfigure}
\renewcommand{\thesubfigure}{\asbuk{subfigure}}